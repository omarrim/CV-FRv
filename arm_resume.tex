\documentclass[]{arm-resume}


\begin{document}


%%%%%%%%%%%%%%%%%%%%%%%%%%%%%%%%%%%%%%
%
%     TITLE NAME
%
%%%%%%%%%%%%%%%%%%%%%%%%%%%%%%%%%%%%%%


\namesection{Omar}{Arrim}{Elève Ingénieur d'Etat \\
Génie Informatique}

%%%%%%%%%%%%%%%%%%%%%%%%%%%%%%%%%%%%%%
%
%     COLUMN ONE
%
%%%%%%%%%%%%%%%%%%%%%%%%%%%%%%%%%%%%%%

\begin{minipage}[t]{0.33\textwidth} 

%%%%%%%%%%%%%%%%%%%%%%%%%%%%%%%%%%%%%%
%     INFORMATIONS
%%%%%%%%%%%%%%%%%%%%%%%%%%%%%%%%%%%%%%

\section{Informations} 


\descript{Né le 20/08/1993}
\location{Mechraa Belksiri | Maroc }
\sectionsep


\descript{Adresse:}
\location{Cité Bouitat | N 01}
Mechraa Belksiri | Maroc\\
\sectionsep

\descript{Mobile \& Fix:}
\location{(+212)0 663 63 94 69 \\ (+212)0 537 90 50 08}
\sectionsep

%%%%%%%%%%%%%%%%%%%%%%%%%%%%%%%%%%%%%%
%     LINKS
%%%%%%%%%%%%%%%%%%%%%%%%%%%%%%%%%%%%%%

\section{Email \& Liens} 
Gmail://  \href{mailto:omar.arrim@gmail.com}{\custombold{omar.arrim}} \\
Yahoo://  \href{mailto:omar.arrim@yahoo.fr}{\custombold{omar.arrim}} \\
Github:// \href{https://github.com/omarrim}{\custombold{omarrim}} \\
LinkedIn://  \href{https://ma.linkedin.com/pub/omar-arrim/51/920/b2}{\custombold{omar-arrim}} \\
\sectionsep


%%%%%%%%%%%%%%%%%%%%%%%%%%%%%%%%%%%%%%
%     SKILLS
%%%%%%%%%%%%%%%%%%%%%%%%%%%%%%%%%%%%%%

\section{Compétences}
\descript{Programmation}
\location{Langages:}
C \textbullet{}   C++ \textbullet{}  C\# \textbullet{}  Java \textbullet{}   PHP \\
 .NET \textbullet{}   J2EE  \textbullet{} JavaScript  \textbullet{} CSS  \\
PL/SQL \textbullet{} Matlab \textbullet{} R \textbullet{}  \LaTeX\ \\ 
\location{Bases de données:}
SGBD: MySQL \textbullet{} ORACLE \\ 
Méthodes : UML \textbullet{} MERISE \\
\location{Divers:}
urbanisation du SI  \textbullet{} Intelligence\\
Artificielle \textbullet{} Gestion de projet  \\
Recherche opérationnelle \\
\sectionsep
\descript{Outils Maîtrisés}
Odoo \textbullet{} Sage 300  \textbullet{} Pentaho DI \\
Wordpress \textbullet{} Bootstrap  \textbullet{} GIT \\
Microsoft Office \textbullet{} SharePoint \\
Adobe Photoshop \textbullet{} RSudio

\sectionsep
\descript{Intérêts}
Data science \textbullet{} Data analysis \\
Business Intelligence \\ Statistique 

\sectionsep

%%%%%%%%%%%%%%%%%%%%%%%%%%%%%%%%%%%%%%
%     LANGUAGES
%%%%%%%%%%%%%%%%%%%%%%%%%%%%%%%%%%%%%%

\section{Langues}
\descript{Arabe} Bilingue\\
\descript{Français} Courant\\
\descript{Anglais} Moyen\\
\descript{Espagnol} Notions\\

\sectionsep

%%%%%%%%%%%%%%%%%%%%%%%%%%%%%%%%%%%%%%
%
%     COLUMN TWO
%
%%%%%%%%%%%%%%%%%%%%%%%%%%%%%%%%%%%%%%

\end{minipage} 
\hfill
\begin{minipage}[t]{0.66\textwidth} 

%%%%%%%%%%%%%%%%%%%%%%%%%%%%%%%%%%%%%%
%     EXPERIENCE
%%%%%%%%%%%%%%%%%%%%%%%%%%%%%%%%%%%%%%

\section{Expériences}

\runsubsection{Atos}
\descript{| Ingénieur Stagiaire d'étude et développement }
\location{Août 2015 – Sep 2015 | Casa NearShore}
\custombold{Project Management Office (PMO):}
\vspace{\topsep}
\begin{tightemize}\item Production et Suivi des indicateurs de performance interne (KPI);
\item Suivi financier du portefolio des projets: budget, revenu (externe et interne), marge réalisée/brute ; mesure des écarts entre coûts budgétés et coûts réels;
\item Optimisation et standardisation des outils de gestion de projet (Excel, développements VBA).\end{tightemize}

\sectionsep

\runsubsection{Maxware technology}
\descript{| Stage d'application }
\location{Juillet 2014 – Août 2014 | Kénitra}
\custombold{Réalisation d’une application de la gestion des marchés:}
\begin{tightemize}
\item Etude préalable et rédaction du cahier des charges de l’application;\item Conception des modéles du systéme d’information (MERISE);\item Implémentation des modules de l’application sur la platforme WINDEV.\end{tightemize}
\sectionsep


%%%%%%%%%%%%%%%%%%%%%%%%%%%%%%%%%%%%%%
%     PROJECTS
%%%%%%%%%%%%%%%%%%%%%%%%%%%%%%%%%%%%%%

\section{Projets}
\location{à ENSA Tétouan}
\runsubsection{FOREX diagram}
\descript{| Cartographie - Data mining/warehouse}
\location{}
Réalisation d'un module de traitement intégré dans un portail du marchés des changes ; programmation d'une stratégie de génération des courbes.
\sectionsep

\runsubsection{Gestion du club sport}
\descript{| Application desktop - java}
\location{}
Conception et réalisation d’une application de la gestion du club sport de l’ENSA Tétouan en utilisant -Modélisation et Programmation orientée objet- UML,JAVA.
\sectionsep

\runsubsection{EnsaBot}
\descript{| Robot - arduino}
\location{}
Création et mise en oeuvre d’un robot mobile simple et économique qui se déplace continuellement tout en évitant les obstacles en se basant sur la carte Arduino.
\sectionsep

\location{à Maxware Technology}
\runsubsection{Gestion des Marchés}
\descript{| Application intranet - windev}
\location{}
Conception et réalisation d’une application de la gestion des marchés répartie sur plusieurs modules avec la platforme WINDEV.
\sectionsep


%%%%%%%%%%%%%%%%%%%%%%%%%%%%%%%%%%%%%%
%     EDUCATION
%%%%%%%%%%%%%%%%%%%%%%%%%%%%%%%%%%%%%%

\section{Formations}

\runsubsection{Génie Informatique}
\descript{| bac+5 }
\location{2013 – 2016 | Ecole Nationale des Scienses Appliquées, Tetouan}


\sectionsep

\runsubsection{Cycle préparatoire}
\descript{| bac+2 }
\location{2011 – 2013 | Ecole Nationale des Scienses Appliquées, Tetouan}

\sectionsep

\runsubsection{Baccalauréat}
\descript{| Sciences Mathématiques A}
\location{2010 – 2011 | Lycée Prince Moulay Rachid, Mechraa Belksiri}

\sectionsep


%%%%%%%%%%%%%%%%%%%%%%%%%%%%%%%%%%%%%%
%     CERTIFICATIONS
%%%%%%%%%%%%%%%%%%%%%%%%%%%%%%%%%%%%%%

\section{Certificats \& CLOM} 

\begin{tabular}{rll}
Aoû 2015   & Coursera   & \href{https://www.coursera.org/learn/big-data-cloud-computing-cdn}{Big Data, Cloud Computing and CDN}\\
& &  \href{https://www.coursera.org/learn/big-data-cloud-computing-cdn}{Emerging Technologies.}\\
& & \custombold{Université Yonsei}\\
Oct 2015 	& Coursera    & \href{https://www.coursera.org/account/accomplishments/certificate/BWM9S9LV5V}{The Data Scientist’s Toolbox.}\\
& & \custombold{Université Johns-Hopkins}\\
Dec 2015 	& Datacamp    & \href{https://www.datacamp.com/courses/free-introduction-to-r-beta}{Introduction to R.}\\

\end{tabular}
\sectionsep

\end{minipage} 
\end{document}  \documentclass[]{article}
